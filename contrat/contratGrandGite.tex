\documentclass[a4paper,11pt]{article}
\usepackage[french]{babel}
\usepackage[T1]{fontenc}
\usepackage[utf8]{inputenc}
\usepackage{lmodern}
\usepackage{microtype}
\usepackage{geometry}
\usepackage{qcm}
\usepackage{eurosym}
\usepackage{alterqcm}

\title{GITE DE L'ECURIEUX\\CONTRAT DE LOCATION SAISONNIERE\\d’un gîte familial écologique\\EVRES EN ARGONNE - MEUSE}
\begin{document}
\date{}
\maketitle

\center Entre les soussignés:\\
\flushleft 
Sonia et Pierre CAILLET\\
2, rue Chatel\\
55100 VERDUN\\                                                                                         
Téléphones portables: 06 50 93 31 87 ou 06 61 30 29 76 \\
Mail: clt.pr@yahoo.com\\
Site: www.ferme-de-l-ecurieux.fr\\
SIRET: 879 952 745 00013\\


\vspace{0.5cm}
Dénommés le bailleur d’une part,                            

\center Et:

\flushleft
Nom:\\
Prénom:\\
Téléphone:\\
Adresse:\\
Pièce d'identité (type et numéro):\\

\vspace{0.5cm}
Dénommé le locataire d'autre part.



\flushleft
Il a été convenu d'une location saisonnière pour la période
\vspace{0.5cm}

du   ...............   à .....    heures .....    (arrivée après 17h)
\vspace{0.5cm}

au   ...............   à .....    heures .....   (départ avant 13h)
\vspace{0.5cm}

Nb de personnes (adultes et enfants\footnote{dormant dans un lit normal}): 
\vspace{0.5cm}

\newpage{}




\vspace{0.5cm}


\begin{tabbing}
  
 \hspace{6cm}\=  										\hspace{1cm}\=   	\hspace{4cm}\= 	\hspace{2cm}\= 		\hspace{2cm}\=	\kill


Montant du loyer de base pour la période: \>		\>		\>=			\>......\>\euro	   		\\


Linge de lit (lits non faits) :			\>5 					\euro		      			\>x ....lit(s)       				\>=  			\>......\>\euro				\\



Linge de toilette (par personne) :					\>2 					\euro      				\> x....							\>=  			\>......\>\euro				\\

Taxe de séjour \>0.7\euro		\>x....pers x...nuits		\>=			\>......\>\euro	   		\\

Total loyer final    										\>						\>  					\>=  							\>......					\>\euro			\\ 

 \end{tabbing}


\vspace{0.5cm}

Nota : la taxe de séjour n’est pas due pour les personnes mineures. Pour les locations
l’utilisant, la plate forme AIRBNB prélève automatiquement.
\vspace{0.5cm}

  
Si vous souhaitez régler par chèque, un chèque d'arrhes de 50\% du loyer final (à l'ordre du bailleur: Sonia et Pierre CAILLET)  doit accompagner le présent contrat correctement rempli et signé. 
Un transfert par virement bancaire est également possible\footnote{IBAN: FR76 1027 8020 0100 0203 2260 118, BIC CMCIFR2A}. Dans ce cas le loyer final et la caution devront être arrivés sur le compte du bailleur avant la date d'entrée dans les lieux.
Dans le cas d'un réglement en espèces à l'entrée dans les lieux, la réception du contrat signé reste absolument nécessaire pour valider la réservation. La caution devra également être présentée en espèces dans ce cas.


\vspace{0.5cm}

A l'entrée dans les lieux, il sera demandé au locataire:
\begin{itemize}
\item le cas échéant, le solde du loyer, par chèque ou en espèces(sauf location AIRBNB).
\item si le paiement s'effectue via AIRBNB, le réglement des éventuels suppléments
\item une caution de 450 euros, par chèque ou en espèces (établissement d'un reçu)
\end{itemize}
En cas de réglement et caution par chèque, ils devront tous être émis par la même personne que le chèque d'arrhes.

\vspace{0.5cm}

Le locataire déclare accepter les conditions générales de location annexées au présent contrat, ainsi qu'avoir pris connaissance de l'état descriptif des lieux.

\vspace{0.5cm}

Fait à ............................ le ...............................................

\vspace{0.5cm}

Le Bailleur : \hspace{3cm}Le Locataire : 

\hspace{5.2cm}mention manuscrite "lu et approuvé" à écrire


\newpage{}




\begin{center}
\section*  { ANNEXE: CONDITIONS GENERALES  DE LOCATION}
\end{center}


\tiny



La présente location est faite aux conditions ordinaires et de droit en pareille matière et notamment à celles ci-après, que le locataire s’oblige à exécuter, sous peine de tous dommages et intérêts et même de résiliations des présentes, si bon semble au mandataire et sans pouvoir réclamer la diminution du loyer.
Le contrat est réputé validé à la réception d’un exemplaire valide et signé du présent contrat et des règlements correspondants.

\begin{enumerate}

\item Les heures d’arrivée et de départ sont inscrites  au contrat par vos soins.
L'entrée dans les lieus se fait après 17h, la sortie avant 13h. Toutefois, des modifications sont possibles par accord mutuel entre le bailleur et le locataire (entrée avant 17h et sortie après 13h).
 


\item Il est interdit de fumer à l’intérieur du gîte. Les mégots des cigarettes fumées à l'extérieur doivent être jetés dans les cendriers.

\item Il est convenu qu'en cas de désistement \\


\begin{itemize}
\item du locataire, quelque soit le motif:
\subitem à plus d'un mois avant la prise d'effet du bail, les arrhes seront reversés au locataire
\subitem à moins d'un mois avant la prise d'effet du bail, la totalité des arrhes seront acquises au bailleur rendant de fait le logement libre
\item du bailleur, quelque soit le motif:
\subitem à plus d'un mois avant la prise d'effet du bail, les arrhes seront reversés au locataire
\subitem à moins d'une semaine avant la prise d'effet du bail, dans les sept jours suivant le désistement, il est tenu de verser le double des arrhes au locataire.
\end{itemize}



\item Si un retard de plus de quatre jours par rapport à la date d’arrivée prévue n’a pas été signalé par le locataire, le bailleur pourra de bon droit, essayer de relouer le logement tout en conservant la faculté de se retourner contre le preneur.



\item Obligation est faite au locataire d’occuper les lieux personnellement, de les habiter “en bon père de famille responsable” et de les entretenir.

\item L'état descriptif des lieux est à rendre dans les deux heures suivant l'arrivée du locataire. Passé ce délai il est réputé accepté et sans remarques.


\item Obligation est faite au locataire de veiller à ce que la tranquillité du voisinage ne soit pas troublée par son fait, celui des personnes qui l'accompagnent ou de son animal domestique.

\item La capacité d'accueil du gîte ne doit pas être dépassée lors des nuits.



\item Les locaux sont loués meublés avec matériel de cuisine, vaisselle, verrerie, draps, couvertures et oreillers (draps en option payante), tels qu’ils sont dans l’état descriptif à vérifier à l'arrivée.
S’il y a lieu, le bailleur ou son représentant seront en droit de réclamer au locataire à son départ et dans les 3 semaines suivantes:\\
- la valeur totale au prix de remplacement des objets, mobiliers ou matériels cassés, fêlés, ébréchés ou détériorés et de ceux dont l’usure dépasserait la normale pour la durée de la location\\
- le prix du nettoyage des couvertures ou couettes rendues sales\\
- une indemnité pour les détériorations de toute nature concernant le billard, les accessoires, les rideaux, papiers peints, plafonds, sols, vitres, literie, etc...\\



\item Le locataire doit être assuré contre les risques locatifs (incendie, dégât des eaux). 
Le bailleur s'engage à assurer le logement contre les risques locatifs pour le compte du locataire, ce dernier ayant l'obligation de lui signaler  tout sinistre survenu dans le logement, ses dépendances ou accessoires avant le départ.

\item L'utilisation du lit superposé supérieur ne convient pas aux enfants de moins de 6 ans.

\item Le locataire déclare accepter le principe d'utilisation de toilettes écologiques à litière bio-maîtrisée.

\item En fonction de l’état des lieux et des déclarations en fin de contrat, le locataire s’engage à régler tous les frais, casse, perte, dégradations etc... résultant du fait de son séjour, y compris ceux découverts après son départ.
Ceux-ci feront l'objet d'une note de frais pouvant être, en toute bonne foi, contestée. 
Le dépôt de garantie (non encore encaissé à cet instant) lui sera alors retourné sous 1 mois maximum, éventuellement diminué des frais évoqués supra.
Dans le cadre des économies d’énergie, le locataire s’engage également à user normalement des énergies (charges gratuites).
Un supplément de frais pourra être facturé en apportant la preuve de l’utilisation anormale.

\item Election de domicile pour tout problème : au domicile du bailleur. Toute action en justice se fera auprès du tribunal de Bar le Duc.

\item Le locataire ne pourra s’opposer à la visite des locaux, lorsque le propriétaire ou son représentant en feront la demande circonstanciée.

\item Les animaux domestiques sont accueillis sous la responsabilité des locataires. Il est interdit au locataire de laisser un animal seul dans le logement. Pour des animaux autres que chiens et chats, l'accord préalable du bailleur est nécessaire.

\end{enumerate}

\hspace{10cm}
Version: \date{\today}


\normalsize

\newpage{}

ETAT DESCRIPTIF DES LIEUX
 
Cet état descriptif est aussi un état des lieux sommaire. Il doit être signé à la remise des clés. En cas de réserve à indiquer, le locataire les indique sur le document avant signature.

\vspace{0.25cm}


Clés
\begin{itemize}
\item deux clés de la porte d'entrée
\item une clé de la porte fenêtre
\end{itemize}


\vspace{0.25cm}

Cour et jardin
\begin{itemize}
\item deux tables forestières, 8 fauteuils, une table de jardin
\item 4 transats, 1 chaises longues, deux parasols
\item une plancha avec bouteille de gaz, détendeur et chariot
\end{itemize}

\vspace{0.25cm}

Cuisine
\begin{itemize}
\item 3 tables de cuisine, 1 banc, 7 chaises, 1 chaise bébé avec tablette
\item frigo avec congélateur, cuisinière gaz, micro ondes, lave vaisselle, bouilloire, cafetière, grille pain
\item buffet avec vaisselle (verres à pied, flûtes, stérilisateur, chauffe biberon, cuiseur vapeur).
\item meubles de cuisine avec vaisselle
\item balai, ensemble de lavage des sols, pelle et brosse sous l'évier
\item téléviseur avec support mural et télécommande
\end{itemize}

\vspace{0.25cm}

Salle billard/piano
\begin{itemize}
\item billard avec deux jeux de boules, triangle, 4 queues, nappe, 2 planches de recouvrement
\item piano avec chaise, jeu de pétanque (8 boules et cochonnet), jeu de quilles finlandaises
\item clic clac et table de nuit, grand miroir mural
\item jeu d'échec africain complet avec deux chaises
\item appareil raclette et fondue pour 8 (NON COMPLET), ensemble couverts barbecue complet, 2 chaises bébé
\item fanion sur support verre, glaive et sabre de décoration
\item RIDEAUX DECHIRES
\end{itemize}

\vspace{0.25cm}

Couloir
\begin{itemize}
\item deux étagères avec livres et CD
\item porte manteau avec glace et support parapluie
\end{itemize}

\vspace{0.25cm}

Salon
\begin{itemize}
\item clic clac et canapé, étagère avec jeux et livres pour enfants
\item chauffeuse, chaise baoulé
\item deux fauteuils osier avec coussin, deux chaises, deux tables basses
\item étagère avec jeux et livres pour enfants
\end{itemize}


\vspace{0.25cm}

Salle de bains
\begin{itemize}
\item lave linge et sèche linge
\item fer et table à repasser, étendoir à linge, étendoir à serviette
\item baignoire, ensemble thermostatique de douche
\item meuble double vasque avec deux miroirs, meuble haut de salle de bains
\end{itemize}

Si cette option a été choisie, le linge de toilette comprend, par personne, une grande serviette et un gant de toilette.
\vspace{0.5cm}

Toilettes
\begin{itemize}
\item lave mains
\item toilette à litière bio maitrisée
\end{itemize}


\vspace{0.25cm}

Dans les chambres, chaque lit dispose d'un matelas avec housse, d'un oreiller avec housse (deux pour les lits doubles),  d'une couverture et d'un couvre lit.

Si cette option a été choisie, le linge de lit comprend, par lit équipé, un drap housse, un drap, une taie d'oreiller (deux pour les lits doubles). 

\vspace{0.5cm}

Petite chambre
\begin{itemize}
\item un lit double, deux lits simples, deux chaises
\item trois tables de nuit, deux lampes de chevet 
\end{itemize}
\vspace{0.25cm}

Chambre parentale
\begin{itemize}
\item un lit double, deux lits simples, deux chaise
\item trois tables de nuit, deux lampes de chevet 
\item armoire 
\item commode, box internet, multiprise anti foudre
\end{itemize}

\vspace{0.5cm}

Grande chambre
\begin{itemize}
\item un lit double, un lit simple, deux chaises
\item lit superposé. Dans les tiroirs: 8 couvertures ou plaids supplémentaires
\item lit bébé avec matelas et mobile musical ABSENT
\item trois tables de nuit, deux lampes de chevet
\item commode
\item armoire (5 sacs de couchage, un mobile musical, deux cartons, lit parapluie)
\item TAPISSERIE ARRACHEE SOUS LA FENETRE
\end{itemize}

\vspace{0.25cm}

L'ensembles des sols, murs et plafond présente un état propre mais usagé, avec diverses traces, mais sans déterioration importante (sauf indiquées supra).

\vspace{0.25cm}

Nombre de lignes annotées à la main: ..........................

\vspace{0.25cm}

Fait à .................................... le ...............................................

\vspace{0.25cm}

Le Bailleur : \hspace{3cm}Le Locataire : 

\hspace{5.2cm}mention manuscrite "lu et approuvé" à écrire


\newpage{}


\end{document}